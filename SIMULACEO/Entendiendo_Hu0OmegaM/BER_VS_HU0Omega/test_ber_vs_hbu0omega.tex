\documentclass[a4paper,10pt]{article}
\usepackage[utf8]{inputenc}

\usepackage{amsmath}
\usepackage{amsfonts}
\usepackage{amssymb}

\usepackage{graphicx}

 
%Para \lstset e insertar codigo
\usepackage{listings}
\usepackage{color}

\lstset{%
  frame=tb,
  language=Octave,%linguagem por defeito
  %
  aboveskip=3mm,
  belowskip=3mm,
  %backgroundcolor=\color{myverylowgray},
  showstringspaces=false,
  columns=flexible,
  basicstyle={\small\ttfamily},
  %
  numbers=none,
  numberstyle=\tiny\color{mygray},
  %
  breaklines=true,
  breakatwhitespace=true,
  tabsize=4
}

%opening
\title{Decodification cost in CEO problem: test\_ber\_vs\_hbu0omega.m}
\author{Fernando Pujaico Rivera}

\begin{document}

\maketitle
%%%%%%%%%%%%%%%%%%%%%%%%%%%%%%%%%%%%%%%%%%%%%%%%%%%%%%%%%%%%%%%%%%%%%%%%%%%%%%%%
\begin{abstract}
test\_ber\_vs\_hbu0omega compare the relation between $P(\hat{U}_0 \neq U_0)$ and
$H(U_0|\Omega_M)$ in the model  system.
\end{abstract}

%%%%%%%%%%%%%%%%%%%%%%%%%%%%%%%%%%%%%%%%%%%%%%%%%%%%%%%%%%%%%%%%%%%%%%%%%%%%%%%%
\section{Introduction}

\lstset{language=octave}%orden importa
\begin{lstlisting}
 test_ber_vs_hbu0omega
\end{lstlisting}
In the Figure \ref{fig:modelo} the source $U_0$, $P(U_0=1)=0.5$, trough across 
$M$ BSC channels, with error probability $P_s$, generating the  sources $U_m$, 
$\forall m \in S=\{1, 2, ..., M\}$. In this context we said $\Omega_M=\{U_m \in S\}$.
In the Figure \ref{fig:modelo2} also can be seen that $\Omega_M=g(U_0)$.
\begin{figure}[h!bt]
\centering
\resizebox {0.9\columnwidth} {!} { \convertMPtoPDF {../modelo.1}{1}{1} }
\caption{System Model (In extend).} \label{fig:modelo}
\end{figure}
Known $\Omega_M$, this values are decoded for to get $\hat{U}_0$, a approximate 
version of $U_0$, so that $\hat{U}_0=f(\Omega_M)$.
\begin{figure}[h!bt]
\centering
\resizebox {0.9\columnwidth} {!} { \convertMPtoPDF {../modelo2.1}{1}{1} }
\caption{System Model (short).} \label{fig:modelo2}
\end{figure}

\begin{equation}
H(U_0|f(\Omega_m)) = H(U_0|\hat{U}_0) = h_b(P(\hat{U_0} \neq U_0))
\end{equation}
thereby
\begin{equation}
h_b(P(\hat{U_0} \neq U_0)) \geq H(U_0|\Omega_M) 
\end{equation}
or if we consider $P(\hat{U_0} \neq U_0)<0.5$
\begin{equation}
P(\hat{U_0} \neq U_0) \geq h_b^{-1}(H(U_0|\Omega_M))
\end{equation}

\subsection{Working with the probability $P(\hat{U_0} \neq U_0)$}
In \cite{Abrardo2009,Ferrari2012} is considered a maximum a posteriori ($MAP$) 
fusion rule $f(\Omega_M)$, 
where the output value $\hat{u}_0$ of $\hat{U}_0$
is obtained as
\begin{equation}
\begin{matrix}
 \hat{u}_0 & =      & \arg_{u_0} \max P(U_0|\Omega_M)\\
 ~         & \equiv & \arg_{u_0} \max P(\Omega_M|U_0)\
\end{matrix}
\end{equation}
Thus, considering that $m_0$ is the number of zeros in $\Omega_M$, the decision is simplify to 
\begin{equation}
m_0
\begin{matrix}
\hat{u}_0=1 \\
\geq \\
< \\
\hat{u}_0=0
\end{matrix}
\lfloor \frac{M}{2}\rfloor
\end{equation}
In this expression is considered that if  $M$ is even and $m_0=M/2$, the decision 
is arbitrarily assume that $\hat{u}_0 = 1$,
so that  $P(\hat{U_0} \neq U_0)$ is $P_e=0.5~[ P(\hat{u}_0=0|u_0=1)+P(\hat{u}_0=1|u_0=0) ]$,
\begin{equation} \label{eq:abrardo}
 P_e= 0.5 \sum_{k=0}^{\lfloor \frac{M}{2} \rfloor -1}  \binom{M}{k} {(1-P_s)}^{k}   {P_s}^{M-k} 
    + 0.5 \sum_{k=\lfloor \frac{M}{2}     \rfloor}^{M} \binom{M}{k} {(1-P_s)}^{M-k} {P_s}^{k}   
\end{equation}
where, $\lfloor . \rfloor$ is the floor function and the value $P_e$ only is valid for
values of $P_s \leq 1/2$ 
\footnote{Here is important note that in \cite{Abrardo2009,Ferrari2012} your value $\rho$
is equal to $1-P_s$ here, and your result is for $\rho > 0.5$}. The Equation 
(\ref{eq:abrardo}) can be sort as
\begin{equation}
P_e= 
\begin{cases}
\sum \limits_{k=\lfloor \frac{M}{2} \rfloor +1}^{M} \binom{M}{k}  (1-P_s)^{M-k} P_s^k & \text{ if } M~odd \\ 
~ & ~ \\
\sum \limits_{k=\lfloor \frac{M}{2} \rfloor +1}^{M} \binom{M}{k}  (1-P_s)^{M-k} P_s^k & \text{ if } M~even \\
~~~~+0.5~\binom{M}{\frac{M}{2}}  (1-P_s)^{\frac{M}{2}} P_s^{\frac{M}{2}} & ~ \\
\end{cases}
\end{equation}
This form is the form showed in \cite{XXXX}.
\subsection{Working with the probability $H(U_0|\Omega_M)$}
\begin{equation}
  H(U_0|\Omega_M)  = \sum_{k=0}^M \binom{M}{k} p_s^k (1-p_s)^{M-k} log_2\left ( 1 +  {\frac{p_s}{(1-p_s)}}^{M-2k} \right )
\end{equation}

 \begin{thebibliography}{99}
\bibitem{Abrardo2009}
Abrardo, A.; Ferrari, G.; Martalò, M.; Perna, F. Feedback Power Control Strategies in 
Wireless Sensor Networks with Joint Channel Decoding. Sensors 2009, 9, 8776-8809.
doi:10.3390/s91108776
 
\bibitem{Ferrari2012}
Ferrari, G.; Martalo, M.; Abrardo, A.; Raheli, R., "Orthogonal multiple access 
and information fusion: How many observations are needed?," Information Theory and 
Applications Workshop (ITA), 2012 , vol., no., pp.311,320, 5-10 Feb. 2012.
doi: 10.1109/ITA.2012.6181783
 
 \end{thebibliography} 

\end{document}
