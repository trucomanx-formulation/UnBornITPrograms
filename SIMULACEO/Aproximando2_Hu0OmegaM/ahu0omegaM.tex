\documentclass[a4paper,10pt]{article}
\usepackage[utf8]{inputenc}

\usepackage{amsmath}
\usepackage{amsfonts}
\usepackage{amssymb}

\usepackage{graphicx}

 
%Para \lstset e insertar codigo
\usepackage{listings}
\usepackage{color}

\lstset{%
  frame=tb,
  language=Octave,%linguagem por defeito
  %
  aboveskip=3mm,
  belowskip=3mm,
  %backgroundcolor=\color{myverylowgray},
  showstringspaces=false,
  columns=flexible,
  basicstyle={\small\ttfamily},
  %
  numbers=none,
  numberstyle=\tiny\color{mygray},
  %
  breaklines=true,
  breakatwhitespace=true,
  tabsize=4
}

%opening
\title{Other complicated way of $H(U_0|U_1 ... U_M)$}
\author{Fernando Pujaico Rivera}

\begin{document}

\maketitle
%%%%%%%%%%%%%%%%%%%%%%%%%%%%%%%%%%%%%%%%%%%%%%%%%%%%%%%%%%%%%%%%%%%%%%%%%%%%%%%%
\begin{abstract}
Prove experimentally one hypothesis.
\end{abstract}

%%%%%%%%%%%%%%%%%%%%%%%%%%%%%%%%%%%%%%%%%%%%%%%%%%%%%%%%%%%%%%%%%%%%%%%%%%%%%%%%
\section{Introduction}

Knowing that:
\begin{equation}
H(U_0|U_1 U_2) = h_b(p_1) + h_b(p_2) - h_b(p_1||p_2)
\end{equation}

where the operator ``$||$'' work the next way:
$a||b \equiv a+b-2ab$. Also $a||b||c \equiv a+b+c-2ab-2ac-2bc+4abc$.
%\lstset{language=octave}%orden importa
%\begin{lstlisting}
% test_ber_vs_hbu0omega
%\end{lstlisting}
It was see experimentally that
\begin{equation}\label{eq:test1}
\begin{matrix}
H(U_0|U_1 U_2 U_3) & \approx & h_b(p_1) + h_b(p_2) + h_b(p_3) \\
~                  & ~       & - h_b(p_1||p_2)- h_b(p_1||p_3) - h_b(p_2||p_3)\\
~                  & ~       & + h_b(p_1||p_2||p_3)\\
\end{matrix}
\end{equation}
and
\begin{equation}\label{eq:test2}
\begin{matrix}
H(U_0|U_1 U_2 U_3 U_4) & \approx & h_b(p_1) + h_b(p_2) + h_b(p_3) + h_b(p_4)\\
~                      & ~       & ~\\
~                      & ~       & - h_b(p_1||p_2)- h_b(p_1||p_3) - h_b(p_1||p_4)\\
~                      & ~       & - h_b(p_2||p_3)- h_b(p_2||p_4) - h_b(p_3||p_4)\\
~                      & ~       & ~\\
~                      & ~       & + h_b(p_1||p_2||p_3)+ h_b(p_1||p_2||p_4)\\
~                      & ~       & + h_b(p_2||p_3||p_4)+ h_b(p_1||p_3||p_4)\\
~                      & ~       & ~\\
~                      & ~       & - h_b(p_1||p_2||p_3||p_4)\\
\end{matrix}
\end{equation}
(\ref{eq:test1}) and (\ref{eq:test2}) were prove experimentally as true.
Them, my  hypothesis is:
\begin{equation}
\begin{matrix}
H(U_0|U_1 ... U_M) & \approx & (-1)^{1+1}~~~\sum \limits_{a_1}        ~~h_b(p_{a_1})~~~~~~~~~~~~~\\
~                  & ~       & (-1)^{2+1}~~\sum \limits_{a_1,a_2}    ~h_b(p_{a_1}||p_{a_2})~~~~~~~\\
~                  & ~       & (-1)^{3+1}\sum \limits_{a_1,a_2,a_3} h_b(p_{a_1}||p_{a_2}||p_{a_3})\\
~                  & ~       & \vdots\\
~                  & ~       & (-1)^{M+1}\sum \limits_{a_1, ...,a_M} h_b(p_{a_1}||...||p_{a_M})\\


\end{matrix}
\end{equation}

\begin{equation}
\begin{matrix}
H(U_1 ... U_M) & \approx & +1\\
~              & ~       & (-1)^{2}~~\sum \limits_{a_1,a_2}    ~h_b(p_{a_1}||p_{a_2})~~~~~~~\\
~              & ~       & (-1)^{3}\sum \limits_{a_1,a_2,a_3} h_b(p_{a_1}||p_{a_2}||p_{a_3})\\
~              & ~       & \vdots\\
~              & ~       & (-1)^{M}\sum \limits_{a_1, ...,a_M} h_b(p_{a_1}||...||p_{a_M})\\


\end{matrix}
\end{equation}

\subsection{Demonstration}

 \begin{thebibliography}{99}
\bibitem{Abrardo2009}
Abrardo, A.; Ferrari, G.; Martalò, M.; Perna, F. Feedback Power Control Strategies in 
Wireless Sensor Networks with Joint Channel Decoding. Sensors 2009, 9, 8776-8809.
doi:10.3390/s91108776
 
\bibitem{Ferrari2012}
Ferrari, G.; Martalo, M.; Abrardo, A.; Raheli, R., "Orthogonal multiple access 
and information fusion: How many observations are needed?," Information Theory and 
Applications Workshop (ITA), 2012 , vol., no., pp.311,320, 5-10 Feb. 2012.
doi: 10.1109/ITA.2012.6181783
 
 \end{thebibliography} 

\end{document}
